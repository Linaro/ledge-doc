%% Generated by Sphinx.
\def\sphinxdocclass{report}
\documentclass[a4paper,10pt,oneside,english]{sphinxmanual}
\ifdefined\pdfpxdimen
   \let\sphinxpxdimen\pdfpxdimen\else\newdimen\sphinxpxdimen
\fi \sphinxpxdimen=.75bp\relax

\PassOptionsToPackage{warn}{textcomp}
\usepackage[utf8]{inputenc}
\ifdefined\DeclareUnicodeCharacter
% support both utf8 and utf8x syntaxes
\edef\sphinxdqmaybe{\ifdefined\DeclareUnicodeCharacterAsOptional\string"\fi}
  \DeclareUnicodeCharacter{\sphinxdqmaybe00A0}{\nobreakspace}
  \DeclareUnicodeCharacter{\sphinxdqmaybe2500}{\sphinxunichar{2500}}
  \DeclareUnicodeCharacter{\sphinxdqmaybe2502}{\sphinxunichar{2502}}
  \DeclareUnicodeCharacter{\sphinxdqmaybe2514}{\sphinxunichar{2514}}
  \DeclareUnicodeCharacter{\sphinxdqmaybe251C}{\sphinxunichar{251C}}
  \DeclareUnicodeCharacter{\sphinxdqmaybe2572}{\textbackslash}
\fi
\usepackage{cmap}
\usepackage[T1]{fontenc}
\usepackage{amsmath,amssymb,amstext}
\usepackage{babel}
\usepackage{times}
\usepackage[Bjarne]{fncychap}
\usepackage[,numfigreset=1,mathnumfig]{sphinx}

\fvset{fontsize=\small}
\usepackage{geometry}

% Include hyperref last.
\usepackage{hyperref}
% Fix anchor placement for figures with captions.
\usepackage{hypcap}% it must be loaded after hyperref.
% Set up styles of URL: it should be placed after hyperref.
\urlstyle{same}

\addto\captionsenglish{\renewcommand{\figurename}{Fig.\@ }}
\makeatletter
\def\fnum@figure{\figurename\thefigure{}}
\makeatother
\addto\captionsenglish{\renewcommand{\tablename}{Table }}
\makeatletter
\def\fnum@table{\tablename\thetable{}}
\makeatother
\addto\captionsenglish{\renewcommand{\literalblockname}{Listing}}

\addto\captionsenglish{\renewcommand{\literalblockcontinuedname}{continued from previous page}}
\addto\captionsenglish{\renewcommand{\literalblockcontinuesname}{continues on next page}}
\addto\captionsenglish{\renewcommand{\sphinxnonalphabeticalgroupname}{Non-alphabetical}}
\addto\captionsenglish{\renewcommand{\sphinxsymbolsname}{Symbols}}
\addto\captionsenglish{\renewcommand{\sphinxnumbersname}{Numbers}}

\addto\extrasenglish{\def\pageautorefname{page}}



\usepackage{draftwatermark}\SetWatermarkScale{.45}\SetWatermarkText{unknown-rev}

\title{LEDGE reference platform user guide}
\date{Nov 06, 2020}
\release{unknown-rev}
\author{Linaro Limited and Contributors}
\newcommand{\sphinxlogo}{\vbox{}}
\renewcommand{\releasename}{Release}
\makeindex
\begin{document}

\pagestyle{empty}
\sphinxmaketitle
\pagestyle{plain}
\sphinxtableofcontents
\pagestyle{normal}
\phantomsection\label{\detokenize{index::doc}}


Copyright © 2020 Linaro Limited and Contributors.

This work is licensed under the Creative Commons Attribution-ShareAlike 4.0
International License. To view a copy of this license, visit
\sphinxurl{http://creativecommons.org/licenses/by-sa/4.0/} or send a letter to
Creative Commons, PO Box 1866, Mountain View, CA 94042, USA.

\sphinxhref{http://creativecommons.org/licenses/by-sa/4.0/}{{\hspace*{\fill}\sphinxincludegraphics{{cc-by-sa-4.0-88x31}.png}}}


\begin{savenotes}\sphinxattablestart
\centering
\sphinxcapstartof{table}
\sphinxthecaptionisattop
\sphinxcaption{Revision History}\label{\detokenize{index:id1}}
\sphinxaftertopcaption
\begin{tabulary}{\linewidth}[t]{l c p{11.5cm}}
\hline
\sphinxstyletheadfamily 
Date
&\sphinxstyletheadfamily 
Issue
&\sphinxstyletheadfamily 
Changes
\\
\hline
17 Febrary 2020
&
0.1
&\begin{itemize}
\item {} 
Initial version

\end{itemize}
\\
\hline
\end{tabulary}
\par
\sphinxattableend\end{savenotes}


\chapter{LEDGE RP USER GUIDE}
\label{\detokenize{user-guide:ledge-rp-user-guide}}\label{\detokenize{user-guide::doc}}

\section{Introduction}
\label{\detokenize{user-guide:introduction}}
This document describes minimal steps to run LEDGE RP precompiled
images in virtual environment and play with it. It’s recommended to
walk over steps in this document for initial introduction with the
Reference Platfrom.  Steps describe howto set up environment and
run binaries and login to the shell. For more technical details and
developer environment refer to LEDGE RP Developer Howto document.

Comments or change requests can be sent to \sphinxhref{mailto:team-ledge@linaro.org}{team-ledge@linaro.org}


\section{Supported platforms}
\label{\detokenize{user-guide:supported-platforms}}\begin{itemize}
\item {} 
armv7/ledge-multi-armv7 (QEMU)

\item {} 
armv8/ledge-multi-armv8 (QEMU)

\item {} 
x86-64 (QEMU)

\end{itemize}


\section{Steps}
\label{\detokenize{user-guide:steps}}

\subsection{Download LEDGE RP binaries:}
\label{\detokenize{user-guide:download-ledge-rp-binaries}}
Download into current directory following files, depending on CPU architecture:

\begin{sphinxVerbatim}[commandchars=\\\{\}]
├── firmware.uefi.uboot.bin
├── firmware.uefi\PYGZhy{}edk2.bin
├── ledge\PYGZhy{}iot\PYGZhy{}ledge\PYGZhy{}qemuarm\PYGZhy{}\PYGZlt{}ts\PYGZgt{}.rootfs.wic.gz
├── ledge\PYGZhy{}kernel\PYGZhy{}uefi\PYGZhy{}certs.ext4.img
\end{sphinxVerbatim}

\begin{DUlineblock}{0em}
\item[] Development daily builds can be found at:
\item[] \sphinxurl{https://snapshots.linaro.org/components/ledge/oe/}
\item[] (Linaro login required)
\item[] Stable releases can be found at:
\item[] \sphinxurl{http://releases.linaro.org/components/ledge/}
\end{DUlineblock}


\subsection{Download QEMU run script:}
\label{\detokenize{user-guide:download-qemu-run-script}}
Download helper script to run QEMU with all required parameters.

\begin{sphinxVerbatim}[commandchars=\\\{\}]
git clone https://git.linaro.org/ledge/scripts.git/
\PYG{n+nb}{cd} scripts/qemu
\end{sphinxVerbatim}


\subsection{Unpack rootfs image:}
\label{\detokenize{user-guide:unpack-rootfs-image}}
\begin{sphinxVerbatim}[commandchars=\\\{\}]
gunzip ledge\PYGZhy{}*rootfs.wic.gz
\end{sphinxVerbatim}


\subsection{Run virtual machine:}
\label{\detokenize{user-guide:run-virtual-machine}}
Depending on your CPU architecture and firmware (UEFI-EDK2 or UEFI-UBOOT)
select one of the following options to run LEDGE RP under virtual machine:
\begin{itemize}
\item {} 
armv7 with TF-A, OP-TEE and U-Boot:
\begin{quote}

./run\_qemu.sh arm ledge-iot-ledge-qemuarm-\textless{}ts\textgreater{}.rootfs.wic
\end{quote}

\item {} 
armv7 with EDK2:
\begin{quote}

./run\_qemu.sh arm ledge-iot-ledge-qemuarm-\textless{}ts\textgreater{}.rootfs.wic ovmf
\end{quote}

\item {} 
armv8 with TF-A, OP-TEE and U-Boot:
\begin{quote}

./run\_qemu.sh aarch64 ledge-iot-ledge-qemuarm64-\textless{}ts\textgreater{}.rootfs.wic
\end{quote}

\item {} 
armv8 with EDK2:
\begin{quote}

./run\_qemu.sh aarch64 ledge-iot-ledge-qemuarm64-\textless{}ts\textgreater{}.rootfs.wic ovmf
\end{quote}

\item {} 
x86\_64 with EDK2:
\begin{quote}

./run\_qemu.sh x86\_64 ledge-iot-ledge-qemux86-64-\textless{}ts\textgreater{}.rootfs.wic ovmf
\end{quote}

\end{itemize}

You should see prints on console that firmware, bootloader, linux boots.


\subsection{Login to the system}
\label{\detokenize{user-guide:login-to-the-system}}
Default user is ‘ledge’ with the same ‘ledge’ password. User is added to
sudoers.



\renewcommand{\indexname}{Index}
\printindex
\end{document}