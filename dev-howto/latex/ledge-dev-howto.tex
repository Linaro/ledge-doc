%% Generated by Sphinx.
\def\sphinxdocclass{report}
\documentclass[a4paper,10pt,oneside,english]{sphinxmanual}
\ifdefined\pdfpxdimen
   \let\sphinxpxdimen\pdfpxdimen\else\newdimen\sphinxpxdimen
\fi \sphinxpxdimen=.75bp\relax

\PassOptionsToPackage{warn}{textcomp}
\usepackage[utf8]{inputenc}
\ifdefined\DeclareUnicodeCharacter
% support both utf8 and utf8x syntaxes
\edef\sphinxdqmaybe{\ifdefined\DeclareUnicodeCharacterAsOptional\string"\fi}
  \DeclareUnicodeCharacter{\sphinxdqmaybe00A0}{\nobreakspace}
  \DeclareUnicodeCharacter{\sphinxdqmaybe2500}{\sphinxunichar{2500}}
  \DeclareUnicodeCharacter{\sphinxdqmaybe2502}{\sphinxunichar{2502}}
  \DeclareUnicodeCharacter{\sphinxdqmaybe2514}{\sphinxunichar{2514}}
  \DeclareUnicodeCharacter{\sphinxdqmaybe251C}{\sphinxunichar{251C}}
  \DeclareUnicodeCharacter{\sphinxdqmaybe2572}{\textbackslash}
\fi
\usepackage{cmap}
\usepackage[T1]{fontenc}
\usepackage{amsmath,amssymb,amstext}
\usepackage{babel}
\usepackage{times}
\usepackage[Bjarne]{fncychap}
\usepackage[,numfigreset=1,mathnumfig]{sphinx}

\fvset{fontsize=\small}
\usepackage{geometry}

% Include hyperref last.
\usepackage{hyperref}
% Fix anchor placement for figures with captions.
\usepackage{hypcap}% it must be loaded after hyperref.
% Set up styles of URL: it should be placed after hyperref.
\urlstyle{same}

\addto\captionsenglish{\renewcommand{\figurename}{Fig.\@ }}
\makeatletter
\def\fnum@figure{\figurename\thefigure{}}
\makeatother
\addto\captionsenglish{\renewcommand{\tablename}{Table }}
\makeatletter
\def\fnum@table{\tablename\thetable{}}
\makeatother
\addto\captionsenglish{\renewcommand{\literalblockname}{Listing}}

\addto\captionsenglish{\renewcommand{\literalblockcontinuedname}{continued from previous page}}
\addto\captionsenglish{\renewcommand{\literalblockcontinuesname}{continues on next page}}
\addto\captionsenglish{\renewcommand{\sphinxnonalphabeticalgroupname}{Non-alphabetical}}
\addto\captionsenglish{\renewcommand{\sphinxsymbolsname}{Symbols}}
\addto\captionsenglish{\renewcommand{\sphinxnumbersname}{Numbers}}

\addto\extrasenglish{\def\pageautorefname{page}}



\usepackage{draftwatermark}\SetWatermarkScale{.45}\SetWatermarkText{unknown-rev}

\title{LEDGE reference platform developer howto}
\date{Nov 06, 2020}
\release{unknown-rev}
\author{Linaro Limited and Contributors}
\newcommand{\sphinxlogo}{\vbox{}}
\renewcommand{\releasename}{Release}
\makeindex
\begin{document}

\pagestyle{empty}
\sphinxmaketitle
\pagestyle{plain}
\sphinxtableofcontents
\pagestyle{normal}
\phantomsection\label{\detokenize{index::doc}}


Copyright © 2020 Linaro Limited and Contributors.

This work is licensed under the Creative Commons Attribution-ShareAlike 4.0
International License. To view a copy of this license, visit
\sphinxurl{http://creativecommons.org/licenses/by-sa/4.0/} or send a letter to
Creative Commons, PO Box 1866, Mountain View, CA 94042, USA.

\sphinxhref{http://creativecommons.org/licenses/by-sa/4.0/}{{\hspace*{\fill}\sphinxincludegraphics{{cc-by-sa-4.0-88x31}.png}}}


\begin{savenotes}\sphinxattablestart
\centering
\sphinxcapstartof{table}
\sphinxthecaptionisattop
\sphinxcaption{Revision History}\label{\detokenize{index:id1}}
\sphinxaftertopcaption
\begin{tabulary}{\linewidth}[t]{l c p{11.5cm}}
\hline
\sphinxstyletheadfamily 
Date
&\sphinxstyletheadfamily 
Issue
&\sphinxstyletheadfamily 
Changes
\\
\hline
17 Febrary 2020
&
0.1
&\begin{itemize}
\item {} 
Initial version

\end{itemize}
\\
\hline
\end{tabulary}
\par
\sphinxattableend\end{savenotes}


\chapter{LEDGE Overview}
\label{\detokenize{overview:ledge-overview}}\label{\detokenize{overview::doc}}

\section{General}
\label{\detokenize{overview:general}}
LEDGE images are related to IoT and EDGE devices. It has advanced security features supported:
\begin{itemize}
\item {} 
Secure UEFI boot

\item {} 
OP-TEE (Open Portable Trusted Execution Environment)

\item {} 
ARM trusted Firmware (AT-F)

\item {} 
TianoCore EDK2 firmware or U-Boot with UEFI mode support

\item {} 
fTPM (Firmware TPM driver with backend to OP-TEE)

\item {} 
Kernel image sign with certificate

\item {} 
Kernel modules sign

\item {} 
IMA/EVM for integrity user applications

\item {} 
SElinux

\item {} 
Containerized isolation (docker)

\item {} 
Advanced system update

\end{itemize}

The LEDGE image consist of WIC image and firmware to boot this image on specific board or virtual machine.
This documentation describes how to build fully open source version of LEDGE reference platform and run it.


\chapter{Build LEDGE RP (OpenEmbedded)}
\label{\detokenize{chapter2-oe:build-ledge-rp-openembedded}}\label{\detokenize{chapter2-oe::doc}}
This chapter describes specific OpenEmbedded LEDGE build and run.


\section{Supported platforms}
\label{\detokenize{chapter2-oe:supported-platforms}}\begin{itemize}
\item {} 
armv7/ledge-multi-armv7 (QEMU, ti-am572x, stm32mp157c-dk2);

\item {} 
armv8/ledge-multi-armv8 (QEMU, synquacer)

\item {} 
x86-64 (QEMU)

\end{itemize}


\section{Build steps}
\label{\detokenize{chapter2-oe:build-steps}}

\subsection{Download sources:}
\label{\detokenize{chapter2-oe:download-sources}}
\begin{sphinxVerbatim}[commandchars=\\\{\}]
repo init \PYGZhy{}\PYGZhy{}no\PYGZhy{}clone\PYGZhy{}bundle \PYGZhy{}\PYGZhy{}depth\PYG{o}{=}\PYG{l+m}{1} \PYGZhy{}\PYGZhy{}no\PYGZhy{}tags \PYGZhy{}u https://github.com/Linaro/ledge\PYGZhy{}oe\PYGZhy{}manifest.git \PYGZhy{}b master
repo sync
\end{sphinxVerbatim}


\subsection{Setup environment and run build:}
\label{\detokenize{chapter2-oe:setup-environment-and-run-build}}

\subsection{armv7 family:}
\label{\detokenize{chapter2-oe:armv7-family}}
\begin{sphinxVerbatim}[commandchars=\\\{\}]
\PYG{n+nv}{MACHINE}\PYG{o}{=}ledge\PYGZhy{}multi\PYGZhy{}armv7 \PYG{n+nv}{DISTRO}\PYG{o}{=}rpb \PYG{n+nb}{source} ./setup\PYGZhy{}environment build\PYGZhy{}rpb
bitbake mc:qemuarm:ledge\PYGZhy{}iot mc:qemuarm:ledge\PYGZhy{}gateway \PYG{l+s+si}{\PYGZdl{}\PYGZob{}}\PYG{n+nv}{FIRMWARE}\PYG{l+s+si}{\PYGZcb{}}
\end{sphinxVerbatim}

Image files will appear under: armhf-glibc/deploy/images directory.

Generated output will be:

\begin{sphinxVerbatim}[commandchars=\\\{\}]
├── ledge\PYGZhy{}qemuarm
│   ├── arm\PYGZhy{}trusted\PYGZhy{}firmware
│   │   ├── bl1.bin
│   │   ├── bl1.elf
│   │   ├── bl2.bin
│   │   └── bl2.elf
│   ├── bl1.bin \PYGZhy{}\PYGZgt{} arm\PYGZhy{}trusted\PYGZhy{}firmware/bl1.bin
│   ├── bl2.bin \PYGZhy{}\PYGZgt{} arm\PYGZhy{}trusted\PYGZhy{}firmware/bl2.bin
│   ├── bl32.bin \PYGZhy{}\PYGZgt{} optee/tee\PYGZhy{}header\PYGZus{}v2.bin
│   ├── bl32\PYGZus{}extra1.bin \PYGZhy{}\PYGZgt{} optee/tee\PYGZhy{}pager\PYGZus{}v2.bin
│   ├── bl32\PYGZus{}extra2.bin \PYGZhy{}\PYGZgt{} optee/tee\PYGZhy{}pageable\PYGZus{}v2.bin
│   ├── bl33.bin \PYGZhy{}\PYGZgt{} u\PYGZhy{}boot\PYGZhy{}ledge\PYGZhy{}qemuarm.bin
│   ├── dtb
│   ├── kernel\PYGZhy{}devicetrees.tgz
│   ├── ledge\PYGZhy{}gateway.env
│   ├── ledge\PYGZhy{}gateway\PYGZhy{}ledge\PYGZhy{}kernel\PYGZhy{}uefi.wks
│   ├── ledge\PYGZhy{}gateway\PYGZhy{}ledge\PYGZhy{}qemuarm\PYGZhy{}20200218104425.bootfs.vfat
│   ├── ledge\PYGZhy{}gateway\PYGZhy{}ledge\PYGZhy{}qemuarm\PYGZhy{}20200218104425.bootfs.vfat.gz
│   ├── ledge\PYGZhy{}gateway\PYGZhy{}ledge\PYGZhy{}qemuarm\PYGZhy{}20200218104425.qemuboot.conf
│   ├── ledge\PYGZhy{}gateway\PYGZhy{}ledge\PYGZhy{}qemuarm\PYGZhy{}20200218104425.rootfs.manifest
│   ├── ledge\PYGZhy{}gateway\PYGZhy{}ledge\PYGZhy{}qemuarm\PYGZhy{}20200218104425.rootfs.wic
│   ├── ledge\PYGZhy{}gateway\PYGZhy{}ledge\PYGZhy{}qemuarm\PYGZhy{}20200218104425.testdata.json
│   ├── ledge\PYGZhy{}gateway\PYGZhy{}ledge\PYGZhy{}qemuarm.bootfs.vfat \PYGZhy{}\PYGZgt{} ledge\PYGZhy{}gateway\PYGZhy{}ledge\PYGZhy{}qemuarm\PYGZhy{}20200218104425.bootfs.vfat
│   ├── ledge\PYGZhy{}gateway\PYGZhy{}ledge\PYGZhy{}qemuarm.bootfs.vfat.gz
│   ├── ledge\PYGZhy{}gateway\PYGZhy{}ledge\PYGZhy{}qemuarm.manifest \PYGZhy{}\PYGZgt{} ledge\PYGZhy{}gateway\PYGZhy{}ledge\PYGZhy{}qemuarm\PYGZhy{}20200218104425.rootfs.manifest
│   ├── ledge\PYGZhy{}gateway\PYGZhy{}ledge\PYGZhy{}qemuarm.qemuboot.conf \PYGZhy{}\PYGZgt{} ledge\PYGZhy{}gateway\PYGZhy{}ledge\PYGZhy{}qemuarm\PYGZhy{}20200218104425.qemuboot.conf
│   ├── ledge\PYGZhy{}gateway\PYGZhy{}ledge\PYGZhy{}qemuarm.testdata.json \PYGZhy{}\PYGZgt{} ledge\PYGZhy{}gateway\PYGZhy{}ledge\PYGZhy{}qemuarm\PYGZhy{}20200218104425.testdata.json
│   ├── ledge\PYGZhy{}gateway\PYGZhy{}ledge\PYGZhy{}qemuarm.wic \PYGZhy{}\PYGZgt{} ledge\PYGZhy{}gateway\PYGZhy{}ledge\PYGZhy{}qemuarm\PYGZhy{}20200218104425.rootfs.wic
│   ├── ledge\PYGZhy{}initramfs\PYGZhy{}ledge\PYGZhy{}qemuarm.cpio.gz \PYGZhy{}\PYGZgt{} ledge\PYGZhy{}initramfs.rootfs.cpio.gz
│   ├── ledge\PYGZhy{}initramfs\PYGZhy{}ledge\PYGZhy{}qemuarm.manifest \PYGZhy{}\PYGZgt{} ledge\PYGZhy{}initramfs.rootfs.manifest
│   ├── ledge\PYGZhy{}initramfs\PYGZhy{}ledge\PYGZhy{}qemuarm.qemuboot.conf \PYGZhy{}\PYGZgt{} ledge\PYGZhy{}initramfs.qemuboot.conf
│   ├── ledge\PYGZhy{}initramfs\PYGZhy{}ledge\PYGZhy{}qemuarm.testdata.json \PYGZhy{}\PYGZgt{} ledge\PYGZhy{}initramfs.testdata.json
│   ├── ledge\PYGZhy{}initramfs.qemuboot.conf
│   ├── ledge\PYGZhy{}initramfs.rootfs.cpio.gz
│   ├── ledge\PYGZhy{}initramfs.rootfs.manifest
│   ├── ledge\PYGZhy{}initramfs.testdata.json
│   ├── ledge\PYGZhy{}iot.env
│   ├── ledge\PYGZhy{}iot\PYGZhy{}ledge\PYGZhy{}kernel\PYGZhy{}uefi.wks
│   ├── ledge\PYGZhy{}iot\PYGZhy{}ledge\PYGZhy{}qemuarm\PYGZhy{}20200218104425.bootfs.vfat
│   ├── ledge\PYGZhy{}iot\PYGZhy{}ledge\PYGZhy{}qemuarm\PYGZhy{}20200218104425.bootfs.vfat.gz
│   ├── ledge\PYGZhy{}iot\PYGZhy{}ledge\PYGZhy{}qemuarm\PYGZhy{}20200218104425.qemuboot.conf
│   ├── ledge\PYGZhy{}iot\PYGZhy{}ledge\PYGZhy{}qemuarm\PYGZhy{}20200218104425.rootfs.manifest
│   ├── ledge\PYGZhy{}iot\PYGZhy{}ledge\PYGZhy{}qemuarm\PYGZhy{}20200218104425.rootfs.wic
│   ├── ledge\PYGZhy{}iot\PYGZhy{}ledge\PYGZhy{}qemuarm\PYGZhy{}20200218104425.testdata.json
│   ├── ledge\PYGZhy{}iot\PYGZhy{}ledge\PYGZhy{}qemuarm.bootfs.vfat \PYGZhy{}\PYGZgt{} ledge\PYGZhy{}iot\PYGZhy{}ledge\PYGZhy{}qemuarm\PYGZhy{}20200218104425.bootfs.vfat
│   ├── ledge\PYGZhy{}iot\PYGZhy{}ledge\PYGZhy{}qemuarm.bootfs.vfat.gz
│   ├── ledge\PYGZhy{}iot\PYGZhy{}ledge\PYGZhy{}qemuarm.manifest \PYGZhy{}\PYGZgt{} ledge\PYGZhy{}iot\PYGZhy{}ledge\PYGZhy{}qemuarm\PYGZhy{}20200218104425.rootfs.manifest
│   ├── ledge\PYGZhy{}iot\PYGZhy{}ledge\PYGZhy{}qemuarm.qemuboot.conf \PYGZhy{}\PYGZgt{} ledge\PYGZhy{}iot\PYGZhy{}ledge\PYGZhy{}qemuarm\PYGZhy{}20200218104425.qemuboot.conf
│   ├── ledge\PYGZhy{}iot\PYGZhy{}ledge\PYGZhy{}qemuarm.testdata.json \PYGZhy{}\PYGZgt{} ledge\PYGZhy{}iot\PYGZhy{}ledge\PYGZhy{}qemuarm\PYGZhy{}20200218104425.testdata.json
│   ├── ledge\PYGZhy{}iot\PYGZhy{}ledge\PYGZhy{}qemuarm.wic \PYGZhy{}\PYGZgt{} ledge\PYGZhy{}iot\PYGZhy{}ledge\PYGZhy{}qemuarm\PYGZhy{}20200218104425.rootfs.wic
│   ├── ledge\PYGZhy{}kernel\PYGZhy{}uefi\PYGZhy{}certs.ext4.img
│   ├── ledge\PYGZhy{}qemuarm.dtb
│   ├── modules\PYGZhy{}ledge\PYGZhy{}qemuarm.tgz \PYGZhy{}\PYGZgt{} modules\PYGZhy{}\PYGZhy{}mainline\PYGZhy{}5.3\PYGZhy{}r0\PYGZhy{}ledge\PYGZhy{}qemuarm\PYGZhy{}20200218104425.tgz
│   ├── modules\PYGZhy{}\PYGZhy{}mainline\PYGZhy{}5.3\PYGZhy{}r0\PYGZhy{}ledge\PYGZhy{}qemuarm\PYGZhy{}20200218104425.tgz
│   ├── modules\PYGZhy{}stripped\PYGZhy{}ledge\PYGZhy{}qemuarm\PYGZhy{}for\PYGZhy{}debian.tgz
│   ├── modules\PYGZhy{}stripped\PYGZhy{}ledge\PYGZhy{}qemuarm.tgz \PYGZhy{}\PYGZgt{} modules\PYGZhy{}stripped\PYGZhy{}\PYGZhy{}mainline\PYGZhy{}5.3\PYGZhy{}r0\PYGZhy{}ledge\PYGZhy{}qemuarm\PYGZhy{}20200218104425.tgz
│   ├── modules\PYGZhy{}stripped\PYGZhy{}\PYGZhy{}mainline\PYGZhy{}5.3\PYGZhy{}r0\PYGZhy{}ledge\PYGZhy{}qemuarm\PYGZhy{}20200218104425.tgz
│   ├── optee
│   │   ├── tee.bin
│   │   ├── tee\PYGZhy{}header\PYGZus{}v2.bin
│   │   ├── tee\PYGZhy{}pageable.bin
│   │   ├── tee\PYGZhy{}pageable\PYGZus{}v2.bin
│   │   ├── tee\PYGZhy{}pager.bin
│   │   └── tee\PYGZhy{}pager\PYGZus{}v2.bin
│   ├── u\PYGZhy{}boot\PYGZhy{}basic\PYGZhy{}1.0\PYGZhy{}r0.bin
│   ├── u\PYGZhy{}boot.bin \PYGZhy{}\PYGZgt{} u\PYGZhy{}boot\PYGZhy{}basic\PYGZhy{}1.0\PYGZhy{}r0.bin
│   ├── u\PYGZhy{}boot.bin\PYGZhy{}basic \PYGZhy{}\PYGZgt{} u\PYGZhy{}boot\PYGZhy{}basic\PYGZhy{}1.0\PYGZhy{}r0.bin
│   ├── u\PYGZhy{}boot\PYGZhy{}ledge\PYGZhy{}qemuarm.bin \PYGZhy{}\PYGZgt{} u\PYGZhy{}boot\PYGZhy{}basic\PYGZhy{}1.0\PYGZhy{}r0.bin
│   ├── u\PYGZhy{}boot\PYGZhy{}ledge\PYGZhy{}qemuarm.bin\PYGZhy{}basic \PYGZhy{}\PYGZgt{} u\PYGZhy{}boot\PYGZhy{}basic\PYGZhy{}1.0\PYGZhy{}r0.bin
│   ├── zImage \PYGZhy{}\PYGZgt{} zImage\PYGZhy{}\PYGZhy{}mainline\PYGZhy{}5.3\PYGZhy{}r0\PYGZhy{}ledge\PYGZhy{}qemuarm\PYGZhy{}20200218104425.bin
│   ├── zImage\PYGZhy{}for\PYGZhy{}debian
│   ├── zImage\PYGZhy{}ledge\PYGZhy{}qemuarm.bin \PYGZhy{}\PYGZgt{} zImage\PYGZhy{}\PYGZhy{}mainline\PYGZhy{}5.3\PYGZhy{}r0\PYGZhy{}ledge\PYGZhy{}qemuarm\PYGZhy{}20200218104425.bin
│   └── zImage\PYGZhy{}\PYGZhy{}mainline\PYGZhy{}5.3\PYGZhy{}r0\PYGZhy{}ledge\PYGZhy{}qemuarm\PYGZhy{}20200218104425.bin
├── ledge\PYGZhy{}stm32mp157c\PYGZhy{}dk2
│   ├── arm\PYGZhy{}trusted\PYGZhy{}firmware
│   │   ├── bl2.bin
│   │   ├── bl2.elf
│   │   └── tf\PYGZhy{}a\PYGZhy{}stm32mp157c\PYGZhy{}dk2.stm32
│   ├── optee
│   │   ├── tee.bin
│   │   ├── tee\PYGZhy{}header\PYGZus{}v2.bin
│   │   ├── tee\PYGZhy{}header\PYGZus{}v2.stm32
│   │   ├── tee\PYGZhy{}pageable.bin
│   │   ├── tee\PYGZhy{}pageable\PYGZus{}v2.bin
│   │   ├── tee\PYGZhy{}pageable\PYGZus{}v2.stm32
│   │   ├── tee\PYGZhy{}pager.bin
│   │   ├── tee\PYGZhy{}pager\PYGZus{}v2.bin
│   │   └── tee\PYGZhy{}pager\PYGZus{}v2.stm32
│   ├── spl
│   │   └── u\PYGZhy{}boot\PYGZhy{}spl.stm32\PYGZhy{}basic
│   ├── u\PYGZhy{}boot\PYGZhy{}basic.img
│   └── u\PYGZhy{}boot\PYGZhy{}trusted.stm32
└── ledge\PYGZhy{}ti\PYGZhy{}am572x
    ├── MLO \PYGZhy{}\PYGZgt{} MLO\PYGZhy{}ledge\PYGZhy{}ti\PYGZhy{}am572x\PYGZhy{}1.0\PYGZhy{}r0
    ├── MLO\PYGZhy{}ledge\PYGZhy{}ti\PYGZhy{}am572x \PYGZhy{}\PYGZgt{} MLO\PYGZhy{}ledge\PYGZhy{}ti\PYGZhy{}am572x\PYGZhy{}1.0\PYGZhy{}r0
    ├── MLO\PYGZhy{}ledge\PYGZhy{}ti\PYGZhy{}am572x\PYGZhy{}1.0\PYGZhy{}r0
    ├── optee
    │   ├── tee.bin
    │   ├── tee\PYGZhy{}header\PYGZus{}v2.bin
    │   ├── tee\PYGZhy{}pageable.bin
    │   ├── tee\PYGZhy{}pageable\PYGZus{}v2.bin
    │   ├── tee\PYGZhy{}pager.bin
    │   └── tee\PYGZhy{}pager\PYGZus{}v2.bin
    ├── u\PYGZhy{}boot.img \PYGZhy{}\PYGZgt{} u\PYGZhy{}boot\PYGZhy{}ledge\PYGZhy{}ti\PYGZhy{}am572x\PYGZhy{}1.0\PYGZhy{}r0.img
    ├── u\PYGZhy{}boot\PYGZhy{}ledge\PYGZhy{}ti\PYGZhy{}am572x\PYGZhy{}1.0\PYGZhy{}r0.img
    └── u\PYGZhy{}boot\PYGZhy{}ledge\PYGZhy{}ti\PYGZhy{}am572x.img \PYGZhy{}\PYGZgt{} u\PYGZhy{}boot\PYGZhy{}ledge\PYGZhy{}ti\PYGZhy{}am572x\PYGZhy{}1.0\PYGZhy{}r0.img
\end{sphinxVerbatim}


\subsection{armv8 family:}
\label{\detokenize{chapter2-oe:armv8-family}}
\begin{sphinxVerbatim}[commandchars=\\\{\}]
\PYG{n+nv}{MACHINE}\PYG{o}{=}ledge\PYGZhy{}multi\PYGZhy{}armv8 \PYG{n+nv}{DISTRO}\PYG{o}{=}rpb \PYG{n+nb}{source} ./setup\PYGZhy{}environment build\PYGZhy{}rpb
bitbake mc:qemuarm64:ledge\PYGZhy{}iot mc:qemuarm64:ledge\PYGZhy{}gateway \PYG{l+s+si}{\PYGZdl{}\PYGZob{}}\PYG{n+nv}{FIRMWARE}\PYG{l+s+si}{\PYGZcb{}}
\end{sphinxVerbatim}


\subsection{x86\_64:}
\label{\detokenize{chapter2-oe:x86-64}}
\begin{sphinxVerbatim}[commandchars=\\\{\}]
\PYG{n+nv}{MACHINE}\PYG{o}{=}ledge\PYGZhy{}qemux86\PYGZhy{}64 \PYG{n+nv}{DISTRO}\PYG{o}{=}rpb \PYG{n+nb}{source} ./setup\PYGZhy{}environment build\PYGZhy{}rpb
bitbake ledge\PYGZhy{}iot ledge\PYGZhy{}gateway
\end{sphinxVerbatim}


\section{Install and boot procedure}
\label{\detokenize{chapter2-oe:install-and-boot-procedure}}\begin{itemize}
\item {} 
DISK=”buildid-rootfs.wic”  - WIC image generated on build procedure. Like ledge-gateway-ledge-qemuarm64-20200216225638.rootfs.wic.

\item {} 
OVMF=”QEMU\_EFI.fd” - OVMF is an EDK II based project to enable UEFI support for Virtual Machines. OVMF contains sample UEFI firmware for QEMU and KVM.

\end{itemize}

OVMF firmware for different architectures can be downloaded from here: \sphinxurl{https://storage.kernelci.org/images/uefi/111bbcf87621/} .

OE maintains script called ‘runqemu’.  This script automatically added to the path after source ./setup-environment is done. This script can be used to run
QEMU virtual machine with all required parameters to boot from image and run networking. Configuration file ledge-iot-ledge-qemuarm-*.qemuboot.conf is
generated during the build process.

Usage example usage:

\begin{sphinxVerbatim}[commandchars=\\\{\}]
runqemu ledge\PYGZhy{}iot\PYGZhy{}ledge\PYGZhy{}qemuarm\PYGZhy{}20200218104425.qemuboot.conf wic serial
\end{sphinxVerbatim}

Example boot log:

\begin{sphinxVerbatim}[commandchars=\\\{\}]
maxim.uvarov@hackbox2:\PYGZti{}/build\PYGZhy{}test\PYGZhy{}update/build\PYGZhy{}rpb\PYGZhy{}mc/armhf\PYGZhy{}glibc/deploy/images/ledge\PYGZhy{}qemuarm\PYGZdl{} runqemu ledge\PYGZhy{}iot\PYGZhy{}ledge\PYGZhy{}qemuarm\PYGZhy{}20200218104425.qemuboot.conf wic serial
runqemu \PYGZhy{} INFO \PYGZhy{} Running \PYG{n+nv}{MACHINE}\PYG{o}{=}ledge\PYGZhy{}qemuarm bitbake \PYGZhy{}e...
runqemu \PYGZhy{} INFO \PYGZhy{} Overriding conf file setting of STAGING\PYGZus{}DIR\PYGZus{}NATIVE to /home/maxim.uvarov/build\PYGZhy{}test\PYGZhy{}update/build\PYGZhy{}rpb\PYGZhy{}mc/tmp\PYGZhy{}rpb\PYGZhy{}glibc/work/armv7at2hf\PYGZhy{}vfp\PYGZhy{}linaro\PYGZhy{}linux\PYGZhy{}gnueabi/defaultpkgname/1.0\PYGZhy{}r0/recipe\PYGZhy{}sysroot\PYGZhy{}native from Bitbake environment
runqemu \PYGZhy{} INFO \PYGZhy{} Continuing with the following parameters:

MACHINE: \PYG{o}{[}ledge\PYGZhy{}qemuarm\PYG{o}{]}
FSTYPE: \PYG{o}{[}wic\PYG{o}{]}
ROOTFS: \PYG{o}{[}/home/maxim.uvarov/build\PYGZhy{}test\PYGZhy{}update/build\PYGZhy{}rpb\PYGZhy{}mc/armhf\PYGZhy{}glibc/deploy/images/ledge\PYGZhy{}qemuarm/ledge\PYGZhy{}iot\PYGZhy{}ledge\PYGZhy{}qemuarm\PYGZhy{}20200218104425.rootfs.wic\PYG{o}{]}
CONFFILE: \PYG{o}{[}/home/maxim.uvarov/build\PYGZhy{}test\PYGZhy{}update/build\PYGZhy{}rpb\PYGZhy{}mc/armhf\PYGZhy{}glibc/deploy/images/ledge\PYGZhy{}qemuarm/ledge\PYGZhy{}iot\PYGZhy{}ledge\PYGZhy{}qemuarm\PYGZhy{}20200218104425.qemuboot.conf\PYG{o}{]}

runqemu \PYGZhy{} INFO \PYGZhy{} Setting up tap interface under sudo
\PYG{o}{[}sudo\PYG{o}{]} password \PYG{k}{for} maxim.uvarov:
runqemu \PYGZhy{} INFO \PYGZhy{} Network configuration: \PYG{l+m}{192}.168.7.2::192.168.7.1:255.255.255.0
runqemu \PYGZhy{} INFO \PYGZhy{} Using block virtio drive
runqemu \PYGZhy{} INFO \PYGZhy{} Interrupt character is \PYG{l+s+s1}{\PYGZsq{}\PYGZca{}]\PYGZsq{}}
runqemu \PYGZhy{} INFO \PYGZhy{} Running sudo /home/maxim.uvarov/build\PYGZhy{}test\PYGZhy{}update/build\PYGZhy{}rpb\PYGZhy{}mc/armhf\PYGZhy{}glibc/work/x86\PYGZus{}64\PYGZhy{}linux/qemu\PYGZhy{}helper\PYGZhy{}native/1.0\PYGZhy{}r1/recipe\PYGZhy{}sysroot\PYGZhy{}native/usr/bin/qemu\PYGZhy{}system\PYGZhy{}arm \PYGZhy{}device virtio\PYGZhy{}net\PYGZhy{}pci,netdev\PYG{o}{=}net0,mac\PYG{o}{=}\PYG{l+m}{52}:54:00:12:34:02 \PYGZhy{}netdev tap,id\PYG{o}{=}net0,ifname\PYG{o}{=}tap0,script\PYG{o}{=}no,downscript\PYG{o}{=}no \PYGZhy{}drive \PYG{n+nv}{id}\PYG{o}{=}disk0,file\PYG{o}{=}/home/maxim.uvarov/build\PYGZhy{}test\PYGZhy{}update/build\PYGZhy{}rpb\PYGZhy{}mc/armhf\PYGZhy{}glibc/deploy/images/ledge\PYGZhy{}qemuarm/ledge\PYGZhy{}iot\PYGZhy{}ledge\PYGZhy{}qemuarm\PYGZhy{}20200218104425.rootfs.wic,if\PYG{o}{=}none,format\PYG{o}{=}raw \PYGZhy{}device virtio\PYGZhy{}blk\PYGZhy{}device,drive\PYG{o}{=}disk0 \PYGZhy{}no\PYGZhy{}reboot \PYGZhy{}show\PYGZhy{}cursor \PYGZhy{}device virtio\PYGZhy{}rng\PYGZhy{}pci \PYGZhy{}monitor null \PYGZhy{}nographic \PYGZhy{}d unimp \PYGZhy{}semihosting\PYGZhy{}config enable,target\PYG{o}{=}native \PYGZhy{}bios bl1.bin \PYGZhy{}dtb ledge\PYGZhy{}qemuarm.dtb \PYGZhy{}drive \PYG{n+nv}{id}\PYG{o}{=}disk1,file\PYG{o}{=}ledge\PYGZhy{}kernel\PYGZhy{}uefi\PYGZhy{}certs.ext4.img,if\PYG{o}{=}none,format\PYG{o}{=}raw \PYGZhy{}device virtio\PYGZhy{}blk\PYGZhy{}device,drive\PYG{o}{=}disk1  \PYGZhy{}machine virt,secure\PYG{o}{=}on \PYGZhy{}cpu cortex\PYGZhy{}a15 \PYGZhy{}m \PYG{l+m}{1024}  \PYGZhy{}device virtio\PYGZhy{}serial\PYGZhy{}device \PYGZhy{}chardev null,id\PYG{o}{=}virtcon \PYGZhy{}device virtconsole,chardev\PYG{o}{=}virtcon

NOTICE:  Booting Trusted Firmware
NOTICE:  BL1: v2.2\PYG{o}{(}debug\PYG{o}{)}:v2.2\PYGZhy{}78\PYGZhy{}g76f25eb52
NOTICE:  BL1: Built : \PYG{l+m}{08}:42:37, Feb \PYG{l+m}{10} \PYG{l+m}{2020}
INFO:    BL1: RAM 0xe04e000 \PYGZhy{} 0xe056000
WARNING: BL1: cortex\PYGZus{}a15: CPU workaround \PYG{k}{for} \PYG{l+m}{816470} was missing!
INFO:    BL1: cortex\PYGZus{}a15: CPU workaround \PYG{k}{for} cve\PYGZus{}2017\PYGZus{}5715 was applied
INFO:    BL1: Loading BL2
WARNING: Firmware Image Package header check failed.
INFO:    Loading image \PYG{n+nv}{id}\PYG{o}{=}\PYG{l+m}{1} at address 0xe01b000
INFO:    Image \PYG{n+nv}{id}\PYG{o}{=}\PYG{l+m}{1} loaded: 0xe01b000 \PYGZhy{} 0xe0201c0
NOTICE:  BL1: Booting BL2
INFO:    Entry point \PYG{n+nv}{address} \PYG{o}{=} 0xe01b000
INFO:    \PYG{n+nv}{SPSR} \PYG{o}{=} 0x1d3
NOTICE:  BL2: v2.2\PYG{o}{(}debug\PYG{o}{)}:v2.2\PYGZhy{}78\PYGZhy{}g76f25eb52
NOTICE:  BL2: Built : \PYG{l+m}{08}:42:37, Feb \PYG{l+m}{10} \PYG{l+m}{2020}
INFO:    BL2: Doing platform setup
INFO:    BL2: Loading image id \PYG{l+m}{4}
WARNING: Firmware Image Package header check failed.
INFO:    Loading image \PYG{n+nv}{id}\PYG{o}{=}\PYG{l+m}{4} at address 0xe100000
INFO:    Image \PYG{n+nv}{id}\PYG{o}{=}\PYG{l+m}{4} loaded: 0xe100000 \PYGZhy{} 0xe10001c
INFO:    OPTEE \PYG{n+nv}{ep}\PYG{o}{=}0xe100000
INFO:    OPTEE header info:
INFO:          \PYG{n+nv}{magic}\PYG{o}{=}0x4554504f
INFO:          \PYG{n+nv}{version}\PYG{o}{=}0x2
INFO:          \PYG{n+nv}{arch}\PYG{o}{=}0x0
INFO:          \PYG{n+nv}{flags}\PYG{o}{=}0x0
INFO:          \PYG{n+nv}{nb\PYGZus{}images}\PYG{o}{=}0x1
INFO:    BL2: Loading image id \PYG{l+m}{21}
WARNING: Firmware Image Package header check failed.
INFO:    Loading image \PYG{n+nv}{id}\PYG{o}{=}\PYG{l+m}{21} at address 0xe100000
INFO:    Image \PYG{n+nv}{id}\PYG{o}{=}\PYG{l+m}{21} loaded: 0xe100000 \PYGZhy{} 0xe12e1f8
INFO:    BL2: Skip loading image id \PYG{l+m}{22}
INFO:    BL2: Loading image id \PYG{l+m}{5}
WARNING: Firmware Image Package header check failed.
INFO:    Loading image \PYG{n+nv}{id}\PYG{o}{=}\PYG{l+m}{5} at address 0x60000000
INFO:    Image \PYG{n+nv}{id}\PYG{o}{=}\PYG{l+m}{5} loaded: 0x60000000 \PYGZhy{} 0x600976bc
NOTICE:  BL1: Booting BL32
INFO:    Entry point \PYG{n+nv}{address} \PYG{o}{=} 0xe100000
INFO:    \PYG{n+nv}{SPSR} \PYG{o}{=} 0x1d3


U\PYGZhy{}Boot \PYG{l+m}{2020}.01 \PYG{o}{(}Feb \PYG{l+m}{10} \PYG{l+m}{2020} \PYGZhy{} \PYG{l+m}{08}:42:58 +0000\PYG{o}{)}

DRAM:  \PYG{l+m}{1} GiB
WARNING: Caches not enabled
Flash: \PYG{l+m}{64} MiB
In:    pl011@9000000
Out:   pl011@9000000
Err:   pl011@9000000
Net:   No ethernet found.
Hit any key to stop autoboot:  \PYG{l+m}{0}
ERROR: reserving fdt memory region failed \PYG{o}{(}\PYG{n+nv}{addr}\PYG{o}{=}7fe00000 \PYG{n+nv}{size}\PYG{o}{=}\PYG{l+m}{200000}\PYG{o}{)}
\PYG{l+m}{1313} bytes \PYG{n+nb}{read} in \PYG{l+m}{2} ms \PYG{o}{(}\PYG{l+m}{640}.6 KiB/s\PYG{o}{)}
Scanning disk virtio\PYGZhy{}blk\PYGZsh{}30...
Scanning disk virtio\PYGZhy{}blk\PYGZsh{}31...
** Unrecognized filesystem \PYG{n+nb}{type} **
Found \PYG{l+m}{4} disks

Warning: virtio\PYGZhy{}net\PYGZsh{}32 using MAC address from ROM
ERROR: reserving fdt memory region failed \PYG{o}{(}\PYG{n+nv}{addr}\PYG{o}{=}7fe00000 \PYG{n+nv}{size}\PYG{o}{=}\PYG{l+m}{200000}\PYG{o}{)}
\PYG{l+m}{2299} bytes \PYG{n+nb}{read} in \PYG{l+m}{1} ms \PYG{o}{(}\PYG{l+m}{2}.2 MiB/s\PYG{o}{)}
ERROR: reserving fdt memory region failed \PYG{o}{(}\PYG{n+nv}{addr}\PYG{o}{=}7fe00000 \PYG{n+nv}{size}\PYG{o}{=}\PYG{l+m}{200000}\PYG{o}{)}
\PYG{l+m}{2299} bytes \PYG{n+nb}{read} in \PYG{l+m}{1} ms \PYG{o}{(}\PYG{l+m}{2}.2 MiB/s\PYG{o}{)}
Booting: kernel
EFI stub: Booting Linux Kernel...
EFI stub: UEFI Secure Boot is enabled.
EFI stub: Using DTB from configuration table
EFI stub: Exiting boot services and installing virtual address map...
\PYG{o}{[}    \PYG{l+m}{0}.000000\PYG{o}{]} Booting Linux on physical CPU 0x0
\PYG{o}{[}    \PYG{l+m}{0}.000000\PYG{o}{]} Linux version \PYG{l+m}{5}.3.6 \PYG{o}{(}oe\PYGZhy{}user@oe\PYGZhy{}host\PYG{o}{)} \PYG{o}{(}gcc version \PYG{l+m}{8}.2.1 \PYG{l+m}{20180802} \PYG{o}{(}Linaro GCC \PYG{l+m}{8}.2\PYGZhy{}2018.08\PYGZti{}dev\PYG{o}{)}\PYG{o}{)} \PYG{c+c1}{\PYGZsh{}1 SMP Tue Feb 18 10:49:14 UTC 2020}
\PYG{o}{[}    \PYG{l+m}{0}.000000\PYG{o}{]} CPU: ARMv7 Processor \PYG{o}{[}412fc0f1\PYG{o}{]} revision \PYG{l+m}{1} \PYG{o}{(}ARMv7\PYG{o}{)}, \PYG{n+nv}{cr}\PYG{o}{=}30c5387d
\PYG{o}{[}    \PYG{l+m}{0}.000000\PYG{o}{]} CPU: div instructions available: patching division code
\PYG{o}{[}    \PYG{l+m}{0}.000000\PYG{o}{]} CPU: PIPT / VIPT nonaliasing data cache, PIPT instruction cache
\PYG{o}{[}    \PYG{l+m}{0}.000000\PYG{o}{]} OF: fdt: Machine model: linux,dummy\PYGZhy{}virt
\PYG{o}{[}    \PYG{l+m}{0}.000000\PYG{o}{]} OF: fdt: Ignoring memory block 0xe00000
\end{sphinxVerbatim}


\subsection{armv7 (qemu\_arm)}
\label{\detokenize{chapter2-oe:armv7-qemu-arm}}
\begin{sphinxVerbatim}[commandchars=\\\{\}]
qemu\PYGZhy{}system\PYGZhy{}arm  \PYG{l+s+se}{\PYGZbs{}}
    \PYGZhy{}device virtio\PYGZhy{}net\PYGZhy{}pci,netdev\PYG{o}{=}net0,mac\PYG{o}{=}\PYG{l+m}{52}:54:00:12:34:02 \PYGZhy{}netdev tap,id\PYG{o}{=}net0,ifname\PYG{o}{=}tap0,script\PYG{o}{=}no,downscript\PYG{o}{=}no \PYG{l+s+se}{\PYGZbs{}}
    \PYGZhy{}drive \PYG{n+nv}{id}\PYG{o}{=}disk0,file\PYG{o}{=}\PYG{l+s+si}{\PYGZdl{}\PYGZob{}}\PYG{n+nv}{DISK}\PYG{l+s+si}{\PYGZcb{}},if\PYG{o}{=}none,format\PYG{o}{=}raw \PYGZhy{}device virtio\PYGZhy{}blk\PYGZhy{}device,drive\PYG{o}{=}disk0 \PYGZhy{}no\PYGZhy{}reboot \PYGZhy{}show\PYGZhy{}cursor \PYG{l+s+se}{\PYGZbs{}}
    \PYGZhy{}device virtio\PYGZhy{}rng\PYGZhy{}pci \PYGZhy{}monitor null \PYGZhy{}nographic \PYG{l+s+se}{\PYGZbs{}}
    \PYGZhy{}d unimp \PYGZhy{}semihosting\PYGZhy{}config enable,target\PYG{o}{=}native \PYGZhy{}bios bl1.bin \PYGZhy{}dtb ledge\PYGZhy{}qemuarm.dtb \PYG{l+s+se}{\PYGZbs{}}
    \PYGZhy{}drive \PYG{n+nv}{id}\PYG{o}{=}disk1,file\PYG{o}{=}ledge\PYGZhy{}kernel\PYGZhy{}uefi\PYGZhy{}certs.ext4.img,if\PYG{o}{=}none,format\PYG{o}{=}raw \PYGZhy{}device virtio\PYGZhy{}blk\PYGZhy{}device,drive\PYG{o}{=}disk1 \PYG{l+s+se}{\PYGZbs{}}
    \PYGZhy{}machine virt,secure\PYG{o}{=}on \PYGZhy{}cpu cortex\PYGZhy{}a15 \PYGZhy{}m \PYG{l+m}{1024}  \PYGZhy{}device virtio\PYGZhy{}serial\PYGZhy{}device \PYG{l+s+se}{\PYGZbs{}}
    \PYGZhy{}chardev null,id\PYG{o}{=}virtcon \PYGZhy{}device virtconsole,chardev\PYG{o}{=}virtcon
\end{sphinxVerbatim}


\subsection{armv8 (qemu\_arm64)}
\label{\detokenize{chapter2-oe:armv8-qemu-arm64}}
OVMF:

\begin{sphinxVerbatim}[commandchars=\\\{\}]
qemu\PYGZhy{}system\PYGZhy{}aarch64 \PYG{l+s+se}{\PYGZbs{}}
   \PYGZhy{}cpu cortex\PYGZhy{}a57 \PYGZhy{}machine virt \PYGZhy{}nographic \PYGZhy{}net nic,model\PYG{o}{=}virtio,macaddr\PYG{o}{=}DE:AD:BE:EF:36:03 \PYGZhy{}net tap \PYGZhy{}m \PYG{l+m}{1024} \PYGZhy{}monitor none \PYG{l+s+se}{\PYGZbs{}}
   \PYGZhy{}bios \PYG{l+s+si}{\PYGZdl{}\PYGZob{}}\PYG{n+nv}{OVMF}\PYG{l+s+si}{\PYGZcb{}} \PYGZhy{}drive \PYG{n+nv}{id}\PYG{o}{=}disk0,file\PYG{o}{=}\PYG{l+s+si}{\PYGZdl{}\PYGZob{}}\PYG{n+nv}{DISK}\PYG{l+s+si}{\PYGZcb{}},if\PYG{o}{=}none,format\PYG{o}{=}raw \PYGZhy{}device virtio\PYGZhy{}blk\PYGZhy{}device,drive\PYG{o}{=}disk0 \PYGZhy{}m \PYG{l+m}{4096} \PYGZhy{}smp \PYG{l+m}{4} \PYGZhy{}nographic
\end{sphinxVerbatim}

U-Boot:

\begin{sphinxVerbatim}[commandchars=\\\{\}]
qemu\PYGZhy{}system\PYGZhy{}aarch64 \PYG{l+s+se}{\PYGZbs{}}
   \PYGZhy{}device virtio\PYGZhy{}net\PYGZhy{}pci,netdev\PYG{o}{=}net0,mac\PYG{o}{=}\PYG{l+m}{52}:54:00:12:34:02 \PYGZhy{}netdev tap,id\PYG{o}{=}net0,ifname\PYG{o}{=}tap0,script\PYG{o}{=}no,downscript\PYG{o}{=}no \PYG{l+s+se}{\PYGZbs{}}
   \PYGZhy{}drive \PYG{n+nv}{id}\PYG{o}{=}disk0,file\PYG{o}{=}\PYG{l+s+si}{\PYGZdl{}\PYGZob{}}\PYG{n+nv}{ROOTFS}\PYG{l+s+si}{\PYGZcb{}},if\PYG{o}{=}none,format\PYG{o}{=}raw \PYGZhy{}device virtio\PYGZhy{}blk\PYGZhy{}device,drive\PYG{o}{=}disk0 \PYGZhy{}show\PYGZhy{}cursor \PYG{l+s+se}{\PYGZbs{}}
   \PYGZhy{}device virtio\PYGZhy{}rng\PYGZhy{}pci \PYGZhy{}monitor null \PYGZhy{}nographic \PYG{l+s+se}{\PYGZbs{}}
   \PYGZhy{}d unimp \PYGZhy{}semihosting\PYGZhy{}config enable,target\PYG{o}{=}native \PYG{l+s+se}{\PYGZbs{}}
   \PYGZhy{}bios bl1.bin \PYG{l+s+se}{\PYGZbs{}}
   \PYGZhy{}drive \PYG{n+nv}{id}\PYG{o}{=}disk1,file\PYG{o}{=}\PYG{l+s+si}{\PYGZdl{}\PYGZob{}}\PYG{n+nv}{KEYS}\PYG{l+s+si}{\PYGZcb{}},if\PYG{o}{=}none,format\PYG{o}{=}raw \PYG{l+s+se}{\PYGZbs{}}
   \PYGZhy{}device virtio\PYGZhy{}blk\PYGZhy{}device,drive\PYG{o}{=}disk1  \PYGZhy{}nographic \PYGZhy{}machine virt,secure\PYG{o}{=}on \PYGZhy{}cpu cortex\PYGZhy{}a57 \PYGZhy{}m \PYG{l+m}{4096} \PYGZhy{}serial mon:stdio \PYGZhy{}serial null \PYG{l+s+se}{\PYGZbs{}}
   \PYGZhy{}no\PYGZhy{}reboot
\end{sphinxVerbatim}


\subsection{x86\_64}
\label{\detokenize{chapter2-oe:id1}}
\begin{sphinxVerbatim}[commandchars=\\\{\}]
qemu\PYGZhy{}system\PYGZhy{}x86\PYGZus{}64 \PYG{l+s+se}{\PYGZbs{}}
   \PYGZhy{}cpu host \PYGZhy{}enable\PYGZhy{}kvm \PYGZhy{}nographic \PYGZhy{}net nic,model\PYG{o}{=}virtio,macaddr\PYG{o}{=}DE:AD:BE:EF:36:03 \PYGZhy{}net tap \PYGZhy{}m \PYG{l+m}{1024} \PYGZhy{}monitor none \PYG{l+s+se}{\PYGZbs{}}
   \PYGZhy{}drive \PYG{n+nv}{file}\PYG{o}{=}\PYG{l+s+si}{\PYGZdl{}\PYGZob{}}\PYG{n+nv}{DISK}\PYG{l+s+si}{\PYGZcb{}},id\PYG{o}{=}hd,format\PYG{o}{=}raw \PYG{l+s+se}{\PYGZbs{}}
   \PYGZhy{}drive \PYG{k}{if}\PYG{o}{=}pflash,format\PYG{o}{=}raw,file\PYG{o}{=}\PYG{l+s+si}{\PYGZdl{}\PYGZob{}}\PYG{n+nv}{OVMF}\PYG{l+s+si}{\PYGZcb{}} \PYG{l+s+se}{\PYGZbs{}}
   \PYGZhy{}m \PYG{l+m}{4096} \PYGZhy{}serial mon:stdio \PYGZhy{}show\PYGZhy{}cursor \PYGZhy{}object rng\PYGZhy{}random,filename\PYG{o}{=}/dev/urandom,id\PYG{o}{=}rng0 \PYGZhy{}device virtio\PYGZhy{}rng\PYGZhy{}pci,rng\PYG{o}{=}rng0
\end{sphinxVerbatim}


\section{Pre built binaries}
\label{\detokenize{chapter2-oe:pre-built-binaries}}
Pre built binaries can be downloaded with the following link:
\sphinxurl{http://snapshots.linaro.org/components/ledge/oe/}
(Linaro account is required).

CI run task can be found here: \sphinxurl{https://ci.linaro.org/job/ledge-oe/}


\chapter{Firmware}
\label{\detokenize{chapter3-firmware:firmware}}\label{\detokenize{chapter3-firmware::doc}}
LEDGE uses 2 \sphinxcite{references:uefi} firmwares:

firmware.uefi-edk2.bin - for EDK2. (Prebuilt binaries are used.)

firmware.uefi.uboot.bin - atf+optee+uefi-uboot. (Firmware compiled during LEDGE build.)


\chapter{LEDGE Internals}
\label{\detokenize{chapter4-internals:ledge-internals}}\label{\detokenize{chapter4-internals::doc}}
This chapter discusses specific features for LEDGE RP.


\section{Applications}
\label{\detokenize{chapter4-internals:applications}}
ledge-iot image package set is alignment with Fedora IoT package set. List of the packages can be found in bitbake recipe ( \sphinxurl{https://github.com/Linaro/meta-ledge/blob/zeus/meta-ledge-sw/recipes-samples/packagegroups/packagegroup-ledge-iot.bb} ).

ledge-gateway image includes minimal console image with OStree and Docker updates support. ( \sphinxurl{https://github.com/Linaro/meta-ledge/blob/zeus/meta-ledge-sw/recipes-samples/images/ledge-gateway.bb} )


\section{U-Boot hardening}
\label{\detokenize{chapter4-internals:u-boot-hardening}}
Security is very important on EDGE nodes. Hardening and protecting the bootloader is the first step towards a secure system. U-boot command line is disabled and kernel boot parameters are present in LEDGE image.


\section{WIC image}
\label{\detokenize{chapter4-internals:wic-image}}
LEDGE WIC image for consists of 2 partitions - ESP partition and linux rootfs partition. Disk image may have
gpt lable type or may not, depends on various hardware requirements.

\begin{sphinxVerbatim}[commandchars=\\\{\}]
Units: sectors of \PYG{l+m}{1} * \PYG{n+nv}{512} \PYG{o}{=} \PYG{l+m}{512} bytes
Sector size \PYG{o}{(}logical/physical\PYG{o}{)}: \PYG{l+m}{512} bytes / \PYG{l+m}{512} bytes
I/O size \PYG{o}{(}minimum/optimal\PYG{o}{)}: \PYG{l+m}{512} bytes / \PYG{l+m}{512} bytes
Disklabel type: gpt
Disk identifier: 5B707C60\PYGZhy{}D281\PYGZhy{}441C\PYGZhy{}A31C\PYGZhy{}73849DD89C49

Device        Start     End Sectors   Size Type
/dev/loop6p1     \PYG{l+m}{40}  \PYG{l+m}{123319}  \PYG{l+m}{123280}  \PYG{l+m}{60}.2M Microsoft basic data
/dev/loop6p2 \PYG{l+m}{123320} \PYG{l+m}{1861731} \PYG{l+m}{1738412} \PYG{l+m}{848}.9M Linux filesystem
\end{sphinxVerbatim}


\subsection{ESP partition}
\label{\detokenize{chapter4-internals:esp-partition}}
ESP partition is about 60 Megabytes vfat partition wit the following structure:

\begin{sphinxVerbatim}[commandchars=\\\{\}]
├── dtb
├── EFI
│   └── BOOT
│       └── bootarm.efi
└── ledge\PYGZhy{}initramfs.rootfs.cpio.gz
\end{sphinxVerbatim}

Where:
\begin{itemize}
\item {} 
dtb - directory which contains all dtbs for all supported devices.

\item {} 
bootarm.efi - linux kernel compiled as UEFI stab and which boots directly from firmware. (bootx64.efi for x86, bootaarch.efi)

\item {} 
ledge-initramfs.rootfs.cpio.gz - initramfs is used to do initial initialization and find and mount rootfs.

\end{itemize}


\section{Run LEDGE RP under QEMU}
\label{\detokenize{chapter4-internals:run-ledge-rp-under-qemu}}
To run LEDGE RP under QEMU the \sphinxcode{\sphinxupquote{run\_qemu.sh}} OpenEmbedded script can be used or the helper QEMU script as described in LEDGE User Guide document.


\section{QEMU with firmware TPM (fTPM) in OP-TEE, TF-A and U-Boot}
\label{\detokenize{chapter4-internals:qemu-with-firmware-tpm-ftpm-in-op-tee-tf-a-and-u-boot}}
LEDGE patches default dtb for QEMU with ftpm entry.

\begin{sphinxVerbatim}[commandchars=\\\{\}]
tpm@0 \PYG{o}{\PYGZob{}}
        \PYG{n+nv}{compatible} \PYG{o}{=} \PYG{l+s+s2}{\PYGZdq{}microsoft,ftpm\PYGZdq{}}\PYG{p}{;}
        linux,sml\PYGZhy{}base \PYG{o}{=} \PYGZlt{}0x0 0xC0000000\PYGZgt{}\PYG{p}{;}
        linux,sml\PYGZhy{}size \PYG{o}{=} \PYGZlt{}0x10000\PYGZgt{}\PYG{p}{;}
\PYG{o}{\PYGZcb{}}\PYG{p}{;}
\end{sphinxVerbatim}

Once qemu is run you might see that tmp\_ftpm\_tee module privides TEE supplicant storage.

\begin{sphinxVerbatim}[commandchars=\\\{\}]
\PYGZti{} \PYG{c+c1}{\PYGZsh{} rmmod tpm\PYGZus{}ftpm\PYGZus{}tee \PYGZam{}\PYGZam{} modprobe tpm\PYGZus{}ftpm\PYGZus{}tee}
\PYGZti{} \PYG{c+c1}{\PYGZsh{} tpm2\PYGZus{}getrandom 256}
\PYG{o}{[}  \PYG{l+m}{107}.057613\PYG{o}{]} audit: \PYG{n+nv}{type}\PYG{o}{=}\PYG{l+m}{1130} audit\PYG{o}{(}\PYG{l+m}{1574348463}.038:33\PYG{o}{)}: \PYG{n+nv}{pid}\PYG{o}{=}\PYG{l+m}{1} \PYG{n+nv}{uid}\PYG{o}{=}\PYG{l+m}{0} \PYG{n+nv}{auid}\PYG{o}{=}\PYG{l+m}{4294967295} \PYG{n+nv}{ses}\PYG{o}{=}\PYG{l+m}{4294967295} \PYG{n+nv}{msg}\PYG{o}{=}\PYG{l+s+s1}{\PYGZsq{}unit=tpm2\PYGZhy{}abrmd comm=\PYGZdq{}systemd\PYGZdq{} exe=\PYGZdq{}/lib/systemd/systemd\PYGZdq{} hostname=? addr=? terminal=? res=success\PYGZsq{}}
0xD0 0x31 0xA5 0xB9 0xF5 0xD1 0x5D 0x91 0x95 0x19 0x59 0x83 0x9F 0x7D 0x12 0x4F 0x8F 0xA2 0x8C 0xC2 0x10 0x71 0x09 0x84 0x6F 0x8B 0x1E 0xE6 0xD4 0xA9 0xA8 0xEB 0xB9 0xAB 0x39 0x92 0x66 0xCB 0x15 0x38 0x7C 0x3F 0x53 0x69 0x86 0xCC 0xA2 0x2A 0x33 0x6B 0x6D 0xFA 0x62 0xC3 0x70 0x93 0x9F 0x96 0xA8 0xFE 0xDA 0x4B 0x4F 0x15
\end{sphinxVerbatim}


\chapter{Terms and abbreviations}
\label{\detokenize{abbreviations:terms-and-abbreviations}}\label{\detokenize{abbreviations::doc}}
This document uses the following terms and abbreviations.
\begin{description}
\item[{A64\index{A64@\spxentry{A64}|spxpagem}\phantomsection\label{\detokenize{abbreviations:term-a64}}}] \leavevmode
The 64-bit Arm instruction set used in AArch64 state.
All A64 instructions are 32 bits.

\item[{AArch32\index{AArch32@\spxentry{AArch32}|spxpagem}\phantomsection\label{\detokenize{abbreviations:term-aarch32}}}] \leavevmode
Arm 32-bit architectures. AArch32 is a roll up term referring to all
32-bit versions of the Arm architecture starting at ARMv4.

\item[{AArch64 state\index{AArch64 state@\spxentry{AArch64 state}|spxpagem}\phantomsection\label{\detokenize{abbreviations:term-aarch64-state}}}] \leavevmode
The Arm 64-bit Execution state that uses 64-bit general purpose
registers, and a 64-bit program counter (PC), Stack Pointer (SP), and
exception link registers (ELR).

\item[{AArch64\index{AArch64@\spxentry{AArch64}|spxpagem}\phantomsection\label{\detokenize{abbreviations:term-aarch64}}}] \leavevmode
Execution state provides a single instruction set, A64.

\item[{EFI Loaded Image\index{EFI Loaded Image@\spxentry{EFI Loaded Image}|spxpagem}\phantomsection\label{\detokenize{abbreviations:term-efi-loaded-image}}}] \leavevmode
An executable image to be run under the UEFI environment,
and which uses boot time services.

\item[{EL0\index{EL0@\spxentry{EL0}|spxpagem}\phantomsection\label{\detokenize{abbreviations:term-el0}}}] \leavevmode
The lowest Exception level on AArch64. The Exception level that is used to execute
user applications, in Non-secure state.

\item[{EL1\index{EL1@\spxentry{EL1}|spxpagem}\phantomsection\label{\detokenize{abbreviations:term-el1}}}] \leavevmode
Privileged Exception level on AArch64. The Exception level that is used to execute
Operating Systems, in Non-secure state.

\item[{EL2\index{EL2@\spxentry{EL2}|spxpagem}\phantomsection\label{\detokenize{abbreviations:term-el2}}}] \leavevmode
Hypervisor Exception level on AArch64. The Exception level that is used to execute
hypervisor code. EL2 is always in Non-secure state.

\item[{EL3\index{EL3@\spxentry{EL3}|spxpagem}\phantomsection\label{\detokenize{abbreviations:term-el3}}}] \leavevmode
Secure Monitor Exception level on AArch64. The Exception level that is used to
execute Secure Monitor code, which handles the transitions between
Non-secure and Secure states.  EL3 is always in Secure state.

\item[{Logical Unit (LU)\index{Logical Unit (LU)@\spxentry{Logical Unit}\spxextra{LU}|spxpagem}\phantomsection\label{\detokenize{abbreviations:term-logical-unit-lu}}}] \leavevmode
A logical unit (LU) is an externally addressable, independent entity
within a device. In the context of storage, a single device may use
logical units to provide multiple independent storage areas.

\item[{OEM\index{OEM@\spxentry{OEM}|spxpagem}\phantomsection\label{\detokenize{abbreviations:term-oem}}}] \leavevmode
Original Equipment Manufacturer. In this document, the final device
manufacturer.

\item[{SiP\index{SiP@\spxentry{SiP}|spxpagem}\phantomsection\label{\detokenize{abbreviations:term-sip}}}] \leavevmode
Silicon Partner. In this document, the silicon manufacturer.

\item[{UEFI\index{UEFI@\spxentry{UEFI}|spxpagem}\phantomsection\label{\detokenize{abbreviations:term-uefi}}}] \leavevmode
Unified Extensible Firmware Interface.

\item[{UEFI Boot Services\index{UEFI Boot Services@\spxentry{UEFI Boot Services}|spxpagem}\phantomsection\label{\detokenize{abbreviations:term-uefi-boot-services}}}] \leavevmode
Functionality that is provided to UEFI Loaded Images during the UEFI boot
process.

\item[{UEFI Runtime Services\index{UEFI Runtime Services@\spxentry{UEFI Runtime Services}|spxpagem}\phantomsection\label{\detokenize{abbreviations:term-uefi-runtime-services}}}] \leavevmode
Functionality that is provided to an Operating System after the
ExitBootServices() call.

\end{description}


\chapter{References}
\label{\detokenize{references:references}}\label{\detokenize{references::doc}}
\begin{sphinxthebibliography}{UEFI}
\bibitem[UEFI]{references:uefi}
\sphinxhref{http://www.uefi.org/sites/default/files/resources/UEFI\%20Spec\%202\_7\_A\%20Sept\%206.pdf}{Unified Extensable Firmware Interface Specification v2.7A},
August 2017, \sphinxhref{http://www.uefi.org}{UEFI Forum}
\end{sphinxthebibliography}



\renewcommand{\indexname}{Index}
\printindex
\end{document}